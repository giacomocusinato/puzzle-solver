%%%%%%%%%%%%%%%%%%%%%%%%%%%%%%%%%%%%%%%%%
% ambiente_di_sviluppo.tex v3.0
%
% Relazione per il progetto PuzzleSolver (Parte 3)
% Autore: Giacomo Cusinato
% Materia: Programmazione Concorrente e Distribuita
%
%%%%%%%%%%%%%%%%%%%%%%%%%%%%%%%%%%%%%%%%%

\section{Ambiente di sviluppo e file di progetto}
Il progetto \`{e} stato realizzato in ambiente Mac OS (versione Yosemite 10.10) sia per la parte relativa al codice Java,
sia per quella relativa alla documentazione.
\`{E} stato inoltre testato con successo nella macchine di laboratorio con i dovuti comandi per impostare Java 7 come
versione principale del sistema.
Per la stesura della relazione, invece, \`{e} stato utilizzato il linguaggio \LaTeX{}  tramite la ditribuzione MacTex.
Come richiesto da specifica, oltre al codice sorgente sono stati inseriti i seguenti file:
\begin{itemize}
    \item \texttt{Makefile}: si occupa di compilare i sorgenti Java
    \item \texttt{puzzlesolverserver.sh}: script bash che si occupa di effettuare il \texttt{make} del progetto
    ed avviare il main dell'applicativo server. Accetta come parametro il nome del server da passare al programma Java.
    \item \texttt{puzzlesolverclient.sh}: script bash che si occupa di effettuare il make del progetto, se necessario,
    ed avviare il main dell'applicativo client. Accetta tre parametri nel seguente ordine: path del file di input, path
    del file di outpute nome del server.
\end{itemize}
Gli script sono stati resi eseguibili tramite il comando bash \texttt{"chmod 754 nomefile"} e possono essere lanciati
semplicemente col comando \texttt{"./puzzlesolver.sh inputPath outputPath"} dalla cartella in cui è presente il file.

\clearpage
