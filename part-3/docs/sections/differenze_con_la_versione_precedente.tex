%%%%%%%%%%%%%%%%%%%%%%%%%%%%%%%%%%%%%%%%%
% differenze_con_la_versione_precedente.tex v2.0
%
% Relazione per il progetto PuzzleSolver (Parte 3)
% Autore: Giacomo Cusinato
% Materia: Programmazione Concorrente e Distribuita
%
%%%%%%%%%%%%%%%%%%%%%%%%%%%%%%%%%%%%%%%%%

\section{Differenze con la versione precedente}

Come da specifica, il progetto è stato suddiviso in modo tale da simulare l'interazione tra un client ed un server
tramite tecnologia RMI. Tutta la logica e le classi relative ala risoluzione del puzzle sono state spostate nella
parte server, in quanto il client si occuperà esclusivamente delle operazioni input/output ottenendo il risultato
finale tramite una chiamata al metodo \texttt{reoder(String inputContent)} definito nell'oggetto remoto.

\subsection{Classi lato server}

Di seguito le classi che definiscono il modulo server del progetto:

\begin{itemize}
    \item \texttt{Block}: rappresenta una tessera del puzzle. Rimane invariata rispetto alle precedenti consegne ma
    fa ora parte del modulo server.
    \item \texttt{PuzzleThread}: La classe PuzzleThread definisce un un oggetto la cui istanza può essere lanciata
    dal programma in un thread a supporto della risoluzione dell’algorimo. Anche questa classe rimane invarita
    rispetto alla precedente consegna ed è stato inserita nel modulo server.
    \item \texttt{PuzzleSolverServer}: è la classe principale del modulo server. Contiene la logica dell'algoritmo di
    risoluzione del puzzle (rimasta invariata) ed espone il metodo \texttt{reoder(String inputContent)} che può essere
    invocato da una JVM remota e restituisce la soluzione prodotta. Il metodo \texttt{main} crea un'istanza di questa
    classe e ne registra il riferimento nel registro RMI.
\end{itemize}


\subsection{Classi lato client}

Di seguito le classi che definiscono il modulo client del progetto:

\begin{itemize}
    \item \texttt{IOHelper}: si occupa di effettura la lettura e la scrittura di stringhe su determinati file.
    Questa classe è rimasta invariata rispetto alle consegne precedenti.
    \item \texttt{PuzzleSolverClient}: è la casse principale del modulo client. Oltre a sfruttare i metodi esposti
    dalla classe \texttt{IOHelper}, si occupa di ottenere il riferimento all'oggetto remoto e chiamare il metodo
    \texttt{reoder(String inputContent)} così da ottenere la soluzione del puzzle tramite il server.
\end{itemize}
