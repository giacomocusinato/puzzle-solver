%%%%%%%%%%%%%%%%%%%%%%%%%%%%%%%%%%%%%%%%%
% abstract.tex v1.0
%
% Relazione per il progetto PuzzleSolver (Parte 1)
% Autore: Giacomo Cusinato
% Materia: Programmazione Concorrente e Distribuita
%
% This document is based on Simple Sectioned Essay template:
% http://www.latextemplates.com/template/simple-sectioned-essay
%
%%%%%%%%%%%%%%%%%%%%%%%%%%		

\section{Abstract}
PuzzleSolver mira ad implementare un risolutore di puzzle, rappresentato come una tabella di tessere, partendo da dei pezzi mischiati. Il puzzle iniziale \`{e} rappresentato da un file di testo tramite una notazione ben precisa e sar\`{a} letto dal programma al suo avvio. La soluzione verr\`{a} infine scritta in un file di output definito dall'utente all'avvio del programma.
Ogni tessera \`{e} identificata univocamente da un attributo di identit\`{a} e conterr\`{a}, inoltre, il testo del carattere che rappresenta e gli identificatori dei pezzi vicini (nord, est, sud ed ovest). Nel caso una tessera si trovi nel bordo del puzzle, la direzione senza pezzi sar\`{a} rappresentata dalla stringa \emph{"'VUOTO'"}.
Il puzzle ha forma matriciale e pu\`{o} assumere qualsiasi dimensione.
