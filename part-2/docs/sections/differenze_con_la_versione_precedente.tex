%%%%%%%%%%%%%%%%%%%%%%%%%%%%%%%%%%%%%%%%%
% scelta_delle_classi.tex v1.0
%
% Relazione per il progetto PuzzleSolver (Parte 1)
% Autore: Giacomo Cusinato
% Materia: Programmazione Concorrente e Distribuita
%
%%%%%%%%%%%%%%%%%%%%%%%%%%%%%%%%%%%%%%%%%

\section{Differenze con la versione precedente}
Il progetto non ha subito molte variazioni rispetto alla versione precedente in quanto le operazioni effettuate dall'algorimo scelto sono
risultate facilmente portabili ad un'architettura concorrente. Di seguito sono riportate le modifiche effettuate alla classe principale,
contentente l'algorimo di risoluzione, ed una semplice descrizione della nuova classe \texttt{PuzzleThread}.



\subsection{La classe \texttt{PuzzleSolver}}
La classe \texttt{PuzzleSolver} è la classe di riferimento del programma, ovvero quella che contiene l'algorimo di risoluzione ed elabora
i dati in accesso ed uscita.

\paragraph{Il metodo \texttt{void reoderRow(int row)}:}
